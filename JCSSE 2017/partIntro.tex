
ในปัจจุบันเรามีการใช้งานอุปกรณ์อิเล็กทรอนิกส์กันมากขึ้น โดยเฉพาะอุปกรณ์พวก smart phone และ tablet จนเราอาจสามารถเรียกได้มันว่าเป็นอวัยวะส่วนหนึ่งของเรา ดังนั้นจึงเป็นเหตุให้มีการพัฒนาโปรแกรมสำหรับใช้งานบนอุปกรณ์พวกนี้ขึ้นเป็นจำนวนมาก

และในการพัฒนาโปรแกรมหนึ่งให้ติดตลาดการใช้งาน ไม่ใช่เพียงแค่ว่าเราพัฒนาโปรแกรมตามความพึงพอใจของเราเพียงอย่างเดียวเท่านั้น แต่เราต้องดูกระแสตอบรับของผู้ใช้งานโปรแกรมนั้น ๆ ด้วย ว่าพวกเขามีความรู้สึกอย่างไรกับโปรแกรมที่เราพัฒนาขึ้นมา มิเช่นนั้นโปรแกรมที่เราพัฒนามานั้นอาจจะไม่มีใครใช้มันเลยก็เป็นไปได้ โดยมีการสำรวจข้อมูลคำถามที่นักพัฒนาต้องการทราบ ซึ่งพบว่านักพัฒนาต้องการทราบ "What parts of a software product are most used and/or loved by customer?" สูงเป็นอับดับที่สอง \cite{145Q}

ดังนั้นใน app store ต่าง ๆ จึงได้มีช่องทางสำหรับให้ผู้ใช้งานมาแสดงถึงปัญหา ความคิดเห็นและให้คะแนนโดยรวมเกี่ยวกับโปรแกรมที่ใช้นั้น ๆ เพื่อให้ผู้ใช้งานคนอื่นที่สนใจ รวมถึงเจ้าของโปรแกรมนั้น ๆ  ได้รับทราบถึงปัญหาหรือคำแนะนำจากผู้ใช้งานคนอื่น ๆ  แต่ทั้งนี้ทั้งนั้นความคิดเห็นของผู้ใช้งานในโปรแกรมนั้น ๆ อาจจะมีจำนวนมากจนทำให้เราไม่สามารถที่จะวิเคราะห์ความคิดเห็นได้ด้วยตนเองทั้งหมด หรืออาจจะทำได้แต่ใช้เวลาที่นานจนทำให้โปรแกรมนั้นอัพเดทหรือปรับปรุงแก้ไขปัญหาที่เกิดได้ไม่ทัน อีกทั้งความคิดเห็นของผู้ใช้งานบางคนอาจจะไม่มีประโยชน์ต่อการวิเคราะห์ข้อมูล (เช่น การบอกว่า "ดี" เพียงอย่างเดียว เราไม่สามารถทราบได้ว่าคำว่าดีที่ว่าหมายถึงอะไร) และ rate ที่ผู้ใช้งานให้มานั้นสามารถบอกได้เพียงแค่ภาพรวมของโปรแกรมเท่านั้น ไม่สามารถแจกแจงได้ว่าส่วนไหนที่ผู้ใช้ชอบหรือไม่ชอบ

ด้วยปัญหาที่กล่าวข้างต้นทำให้เกิดงานวิจัยที่ใช้ในการวิเคราะห์ความคิดเห็นของผู้ใช้งานโปรแกรมเหล่านี้เป็นจำนวนมาก โดยมีเป้าหมายในการช่วยลดภาระการวิเคราะห์ความคิดเห็นของผู้ใช้งาน ไม่ว่าจะเป็นการหาข้อมูลที่มีสาระประโยชน์จากความคิดเห็นทั้งหมด หรือการสกัดเอาคำสำคัญของความคิดเห็นเหล่านั้นขึ้นมาเพื่อจับกลุ่มของแสดงหัวข้อที่ผู้ใช้งานกล่าวถึง

แต่ด้วยในขณะนี้งานวิจัยที่มียังไม่สามารถนำมาใช้กับการวิเคราะห์ความคิดเห็นภาษาไทยได้  ซึ่งงานวิจัยนี้จึงมีจุดประสงค์ในการพัฒนากระบวนการและแนวคิดในการวิเคราะห์ความคิดเห็นที่เป็นภาษาไทยเพื่อเป็นแนวทางในการวิเคราะห์ข้อมูลได้อย่างมีประสิทธิภาพมากขึ้น โดยในหัวข้อ \ref{Background} จะกล่าวถึงทฤษฎีที่มีความสำคัญเกี่ยวกับงานวิจัยนี้, หัวข้อ \ref{Related Work} กล่าวถึงงานวิจัยที่เกี่ยวข้อง, หัวข้อ \ref{Approach} กระบวนการในการวิจัย, หัวข้อ \ref{Result} ผลลัพธ์ของการวิจัย และหัวข้อ \ref{Conclusion} จะเป็นการสรุปงานวิจัยนี้
