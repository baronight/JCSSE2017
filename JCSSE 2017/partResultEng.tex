We evaluated our approach by asking one person not related our work to manually label sentiments for each sentence whether it is positive, negative, or neutral. This information is used as truth values to calculate precision, recall, and accuracy. Table~\ref{table:f-measureSenti} shows evaluation results.

\begin{table}[h]
	\caption{F-Measure and Accuracy for Sentiment Analysis}
	\label{table:f-measureSenti}
	\centering
	\begin{tabular}{|l|r|r|r|r|}
		\hline
		\multicolumn{1}{|c|}{\textbf{Application}} &
		\multicolumn{1}{|c|}{\textbf{Precision}} &
		\multicolumn{1}{|c|}{\textbf{Recall}} &
		\multicolumn{1}{|c|}{\textbf{F-Measure}} &
		\multicolumn{1}{|c|}{\textbf{Accuracy}} \\
		\hline
		Man Man & 0.5028 & 0.3189 & 0.3570 & 0.6110\\
		\hline
		H-TV & 0.5208 & 0.2889 & 0.3366 & 0.4837 \\
		\hline
		K-Mobile & 0.4535 & 0.2810 & 0.3240 & 0.5153 \\
		\hline
	\end{tabular}
\end{table}

\subsection*{Limitation}
Our work still has some limetaions. Some limitations are mentioned in Sention~\ref{Approach} such as limitation in sentence and word segmentations of information texts, slangs, and misspelled words. This leads the LEXiTRON not able to find word translation and SentiWordNet not able to find sentiment scores. In addition, some words have several meanings and LEXiTRON returns several translation. We therefore would not get precise meanings and therefore not getting precise sentiment score.

For topic modelling, our work fixed a number of topics, and therefore not too automatic. It should be better if the approach can specify the number of topics dynamically to match different mobile applications. We can estimate the number of topics from a number of features listed on the mobile application websites.

Ask more human to label data to ...


% An example of a double column floating figure using two subfigures.
% (The subfig.sty package must be loaded for this to work.)
% The subfigure \label commands are set within each subfloat command,
% and the \label for the overall figure must come after \caption.
% \hfil is used as a separator to get equal spacing.
% Watch out that the combined width of all the subfigures on a 
% line do not exceed the text width or a line break will occur.
%
%\begin{figure*}[!t]
%\centering
%\subfloat[Case I]{\includegraphics[width=2.5in]{box}%
%\label{fig_first_case}}
%\hfil
%\subfloat[Case II]{\includegraphics[width=2.5in]{box}%
%\label{fig_second_case}}
%\caption{Simulation results for the network.}
%\label{fig_sim}
%\end{figure*}
%
% Note that often IEEE papers with subfigures do not employ subfigure
% captions (using the optional argument to \subfloat[]), but instead will
% reference/describe all of them (a), (b), etc., within the main caption.
% Be aware that for subfig.sty to generate the (a), (b), etc., subfigure
% labels, the optional argument to \subfloat must be present. If a
% subcaption is not desired, just leave its contents blank,
% e.g., \subfloat[].


% An example of a floating table. Note that, for IEEE style tables, the
% \caption command should come BEFORE the table and, given that table
% captions serve much like titles, are usually capitalized except for words
% such as a, an, and, as, at, but, by, for, in, nor, of, on, or, the, to
% and up, which are usually not capitalized unless they are the first or
% last word of the caption. Table text will default to \footnotesize as
% the IEEE normally uses this smaller font for tables.
% The \label must come after \caption as always.
%
%\begin{table}[!t]
%% increase table row spacing, adjust to taste
%\renewcommand{\arraystretch}{1.3}
% if using array.sty, it might be a good idea to tweak the value of
% \extrarowheight as needed to properly center the text within the cells
%\caption{An Example of a Table}
%\label{table_example}
%\centering
%% Some packages, such as MDW tools, offer better commands for making tables
%% than the plain LaTeX2e tabular which is used here.
%\begin{tabular}{|c||c|}
%\hline
%One & Two\\
%\hline
%Three & Four\\
%\hline
%\end{tabular}
%\end{table}


%\begin{table*}[h]
%	\caption{example of result after pass POS tagger process}
%	\label{table:POSEx}
%	\centering
%	\begin{tabular}{|l|l|l|}
%		\hline
%		\multicolumn{1}{|c|}{sentense} &
%		\multicolumn{1}{|c|}{word segmentation} &
%		\multicolumn{1}{|c|}{POS}\\
%		\hline
%		ใช้ได้ดีครับ & 
%		ใช้ได้|ดี|ครับ| & 
%		ใช้ได้/npn ดี/vi ครับ/aff \\
%		\hline
%		เยี่ยม ดี เลิศ & 
%		เยี่ยม|ดี|เลิศ| & 
%		เยี่ยม/vt ดี/adv เลิศ/adv \\
%		\hline
%		พักหลังนี่อัพบ่อยนะครับ & 
%		พัก|หลัง|นี่|อัพบ่อย|นะ|ครับ| & 
%		พัก/vi หลัง/adj นี่/pdem อัพบ่อย/npn นะ/part ครับ/aff \\
%		\hline
%		ชอบค่ะใช้ง่าย มีตัวการ์ตูนให้ด้วย & 
%		ชอบ|ค่ะ|ใช้|ง่าย|มี|ตัว|การ์ตูน|ให้|ด้วย| & 
%		ชอบ/vt ค่ะ/aff ใช้/vt ง่าย/adv มี/vt ตัว/ncn การ์ตูน/ncn ให้/vpost ด้วย/adv \\
%		\hline
%		เรียบง่ายแต่ใช้ได้ดีจริงๆครับชอบมาก & 
%		เรียบง่าย|แต่|ใช้ได้|ดี|จริงๆ|ครับ|ชอบมาก| & 
%		เรียบง่าย/vi แต่/conj ใช้ได้/npn ดี/vi จริงๆ/adv ครับ/aff ชอบ/vt มาก/adv \\
%		\hline
%		ดีมากครับ สะดวกดีแม่นสุดยอด & 
%		ดีมาก|ครับ|สะดวก|ดี|แม่น|สุดยอด| & 
%		ดีมาก/npn ครับ/aff สะดวก/vi ดี/adv แม่น/vt สุดยอด/adj \\
%		\hline
%	\end{tabular}
%\end{table*}

