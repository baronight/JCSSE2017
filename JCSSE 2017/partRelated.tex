
เราได้แบ่งงานวิจัยที่เกี่ยวข้องออกเป็น 2 กลุ่ม คือ 1. งานวิจัยที่เกี่ยวข้องกับการวิเคราะห์ทัศนคติของผู้ใช้งานโปรแกรมบนโทรศัพท์เคลื่อนที่ และ 2. งานวิจัยที่เกี่ยวข้องกับการวิเคราะห์ข้อมูลภาษาไทย

\subsubsection{งานวิจัยที่เกี่ยวข้องกับการวิเคราะห์ทัศนคติของผู้ใช้งานโปรแกรมบนโทรศัพท์}
จากการศึกษาพบว่างานวิจัยทางด้านการวิเคราะห์ทัศนคติของผู้ใช้งานโปรแกรมบนโทรศัพท์เคลื่อนที่นั้นมีจำนวนมาก แต่ส่วนใหญ่จะมีแต่งานวิจัยที่วิเคราะห์ข้อมูลภาษาอังกฤษเพียงอย่างเดียว \cite{ar-miner,keywordmining,userslikefeature} เป็นเหตุให้งานวิจัยเหล่านี้อาจไม่สามารถนำมาใช้ในการวิเคราะห์ข้อมูลภาษาอื่น ๆ ได้

Ning Chen และคณะ\cite{ar-miner} ได้นำเสนอ AR-Miner เป็นงานวิจัยที่จะสกัดและจัดอันดับประโยคที่มีสารประโยชน์ (informative reviews) จากข้อความของผู้ใช้งาน โดยพวกเขาใช้วิธีการแบ่งกลุ่มคำที่มีประโยชน์และไม่มีประโยชน์ (classifier) ด้วยวิธี Expectation Maximization for Naive Bays (EMNB) \cite{EMNB} ซึ่งเป็นการจัดกลุ่มโดยการนำข้อมูลที่ทราบคำตอบมาเป็นแบบ ในการหาข้อมูลที่ไม่ทราบคำตอบ จากนั้นจึงจะนำข้อมูลที่เป็นประโยชน์มาแบ่งกลุ่มตามหัวข้อที่ผู้ใช้ได้กล่าวถึง โดยการเปรียบเทียบระหว่างวิธีการ LDA และ ASUM

Emitza Guzmann และ Wiem Maalej \cite{userslikefeature} ได้เสนอวิธีการสกัดหา feature และ sentiment ของโปรแกมจาก review ของผู้ใช้งาน โดยพวกเขาจะใช้เฉพาะ noun, verb, and adjective ที่อยู่ในประโยคมาแทน feature และใช้ SentiStrengh \cite{SentiStrength} ในการหา sentiment ของแต่ละคำ แล้วนำคะแนนที่มากที่สุดของคำในประโยคที่มาเป็นคะแนนของประโยค และใช้ LDA ในการจับกลุ่ม feature ต่าง ๆ ซึ่งในการวัดความถูกต้องนั้นพวกเขาได้ให้นักวิจัยอีกคนที่ไม่มีส่วนเกี่ยวกับการทำงานวิจัยนี้มาเปรียบเทียบ

Phong Minh Vu และคณะ \cite{keywordmining} ได้เสนอวิธีการหาประโยคที่ผู้ใช้งานได้กล่าวต่อว่าหรือกล่าวถึงปัญหาของโปรแกรม จากการหา keyword ของคำในประโยค โดย keyword ที่พวกเขาใช้จะเป็นคำ noun และ verb และจับกลุ่ม keyword เหล่านี้ด้วยการหาค่าความใกล้เคียงของคำด้วยวิธี cosine similarity \cite{cosinesimilarity} ซึ่งพวกเขาเปรียบเทียบความถูกต้องของงานวิจัยโดยการให้ผู้เชี่ยวชาญ 8 คนตรวจสอบความเหมาะสมของกลุ่มคำสำคัญที่หาได้

โดยงานวิจัยเหล่านี้มีความคล้ายกับงานวิจัยนี้ตรงที่ต่างก็ต้องการหาวิธีที่จะแสดงถึงหัวข้อที่ผู้ใช้งานต้องการสื่อสาร โดยเฉพาะอย่างยิ่งงานวิจัยของ Emitza ที่มีการวิเคราะห์หา sentiment แต่ทั้งนี้ทั้งนั้นงานวิจัยเหล่านี้อยู่เป็นการวิจัยที่ทำอยู่บนฐานของภาษาอังกฤษ ซึ่งสำหรับการวิเคราะห์ภาษาไทยนั้นจะมีความแตกต่างกับการวิเคราะห์ภาษาอังกฤษอยู่บางส่วน


\subsubsection{งานวิจัยที่เกี่ยวข้องกับการวิเคราะห์ข้อมูลภาษาไทย}
จากการศึกษาพบว่ามีงานวิจัยที่เกี่ยวกับการวิเคราะห์ข้อมูลภาษาไทยอยู่จำนวนหนึ่ง ซึ่งมีทั้ง การสรุปบทความของเอกสารจากย่อหน้า \cite{paragraphextract}, การจัดกลุ่มคำตามอารมณ์พื้นฐานของมนุษย์ \cite{emotioninthai}, การวิเคราะห์ทัศนคติบนสื่อสังคมออนไลน์ \cite{ssense}, และการวิเคราะห์ทัศนคติของผู้ใช้งานโรงแรม \cite{thaiopinionmininghotel} แต่ผู้วิจัยยังไม่พบงานวิจัยที่เกี่ยวกับการวิเคราะห์ทัศนคติของโปรแกรมบนโทรศัพท์

Chuleerat Jaruskulchai และ Canasai Kruengkrai \cite{paragraphextract} ได้เสนอวิธีการสรุปบทความของเอกสารภาษาไทยด้วยการสกัดหาย่อหน้าที่สำคัญ โดยพวกเขาได้แบ่งข้อความออกเป็นย่อหน้า และตัดคำของแต่ละย่อหน้าด้วยวิธีการ Longest matching จากนั้นจึงนำคำที่ตัดได้ไปหาความสัมพันธ์ระหว่างแต่ละย่อหน้า เพื่อเชื่อมโยงย่อหน้าที่มีความเกี่ยวข้องกัน จากคำที่ต้องการค้นหา โดยใช้นักศึกษาจากคณะศิลปศาสตร์ สาขาภาษาไทย สรุปเอกสารเหล่านี้เพื่อใช้วัดความถูกต้องของงานวิจัย

Piyatida และ Sukree Sinthupinyo \cite{emotioninthai} ได้เสนอวิธีการจัดกลุ่มคำที่แสดงถึงอารมณ์ จากคำนามและคำกริยา โดยใช้ SWATH ตัดคำซึ่งเป็นโปรแกรมที่ใช้ในการตัดคำที่พัฒนาโดย NECTEC และใช้ ORCHID corpus เพื่อหา POS ของคำเหล่านั้น แล้วนำคำที่เป็นคำนามและกริยามาจัดกลุ่มตามอารมณ์พื้นฐานของมนุษย์ 6 อย่างคือ โกรธ, ขยะแขยง, กลัว, ดีใจ เสียใจ, และตกใจ \cite{basicemotion} ด้วยวิธีการจัดกลุ่มแบบ Naive Bays 

Choochart Haruechaiyasak และคณะ \cite{ssense} ได้เสนอ S-Sense เป็นเครื่องมือสำหรับการวิเคราะห์ทัศนคติบนสื่อสังคมออนไลน์ เช่น twitter และ pantip (webboard ที่แพร่หลายของคนไทย) เป็นต้น โดยใช้ LEXiTRON ซึ่งเป็นพจนานุกรมไทย-อังกฤษ และอังกฤษ-ไทย ที่พัฒนาโดย NECTEC เพื่อแปลภาษาและจับกลุ่มของคำที่มีความหมายใกล้เคียงกัน และใช้ Utilization on REsource for Knowledge Acquisition (UREKA) ซึ่งเป็นส่วนประกอบส่วนหนึ่งของงานวิจัยของเขา ในการสกัดหัวข้อ/คำสำคัญ ออกมาจากข้อความ โดยงานวิจัยนี้จะแบ่งกลุ่มของข้อความออกเป็น การประกาศ การแสดงความต้องการ การแสดงคำถาม และความคิดเห็น โดยพวกเขาได้แยกความคิดเห็นออกเป็น เชิงบวก และ เชิงลบ

Choochart Haruechaiyasak และคณะ \cite{thaiopinionmininghotel} ได้เสนอวิธีการวิเคราะห์ทัศนคติของผู้ที่มาใช้โรงแรม โดยใช้ lexicon และ corpus สำหรับค้นหาคำและทัศนคติของคำ โดยงานวิจัยนี้มีหัวข้อสำหรัยการวิเคราะห์ทัศนคติที่ชัดเจนคือ 1. การบริการ, และ 2. อาหารเช้า โดยการศึกษาหา pattern ของประโยค เพื่อจับกลุ่มประโยคให้อยู่ในหัวข้อที่ต้องการ และนำกลุ่มหัวข้อเหล่านั้นไปวิเคราะห์ทัศนคติต่อไป

โดยงานวิจัยเหล่านี้ต่างก็ต้องมีการหา POS เพื่อที่จะกรองชนิดของคำที่ต้องการนำมาวิจัย (ซึ่งส่วนมากจะเป็น noun, verb, adjective) โดยงานวิจัย S-Sense และ การหาทัศนคติของผู้ใช้งานโรงแรม ต่างมีความต้องการที่จะหาทัศนคติของผู้ใช้งานเหมือนงานวิจัยฉบับนี้ แต่ก็มีความแตกต่างกันอยู่บ้างตรงที่ การหาทัศนคติของผู้ใช้งานโรงแรม นั้นจะมีหัวข้อที่ต้องการจะหาอยู่แล้วอย่างแน่นอน และของ S-Sense จะสามารถใช้กับสื่อสังคมออนไลน์ได้ แต่ไม่สามารถนำมาใช้กับการวิเคราะห์โปรแกรมที่อยู่ในโทรศัพท์ ได้
ซึ่งสำหรับงานวิจัยฉบับนี้ ต้องการที่จะวิเคราะห์หาหัวข้อและทัศนคติของโปรแกรมที่อยู่ในโทรศัพท์ และในแต่ละโปรแกรมก็จะมีหัวข้อที่ถูกกล่าวถึงต่างกัน ทำให้ไม่สามารถที่จะกำหนดหัวข้อที่ต้องการจะวิเคราะห์ได้อย่างชัดเจน

