There are a number of researches on opinion mining and sentiment analysis related to our work. This section describes two groups of researches. The first group is on opinion mining and sentiment analysis of user reviews of mobile applications. The second group is on mining of Thai texts. 

\subsubsection{Research on opinion mining and sentiment analysis of user reviews of mobile applications} Various researches are in this group. Most of them analyzed user reviews written in English on mobile applications. We will discuss three of these works \cite{ar-miner,userslikefeature,keywordmining}. Another work on evaluating software quality in use from user reviews \cite{leopairote2} is also discussed here.

Chen et al.\cite{ar-miner} presented an AR-Miner framework which filtered reviews and extracted only informative reviews using a classification technique called Expectation Maximization for Naive Bays (EMNB) \cite{EMNB}, which is a semi-supervised machine learning algorithm. The resulting informative reviews were then grouped into topics. The work applied two topic modeling techniques, Latent Dirichlet Allocation (LDA) \cite{LDA} and Aspect and Sentiment Unification Model (ASUM) ~\cite{asum}, and compared results of these techniques.

Gunzmam and Laalej \cite{userslikefeature} analyzed user reviews to identify features and extract their associated sentiments. Identifying features were done by finding expressions of two or more words that commonly occurred together such as <\textit{view picture}>, or <\textit{user interface}>. User sentiments were analyzed using SentiStrength \cite{SentiStrength} for lexical sentiment extraction. Finally, LDA was used to group various features into coherent topics. The approach was evaluated with precision and recall where the truth set was created by the two authors and other seven trained coders who manually analyzed user reviews. 

Minh et al. \cite{keywordmining} presented a semi-automated framework for mining user reviews when given keywords specified by analysts. The approach can also automatically extract keywords from nouns and verbs in user reviews. The keywords were clustered and expanded and then used to search for relevant reviews hopefully containing useful opinions. The evaluation was done by comparing analysis results with eight researchers.

Leopairote et al. \cite{leopairote2} evaluated software quality in use by performing opinion mining of user reviews. Their approach constructed an ontology from the quality in use model, which is one of software quality models in ISO 9126, consisting of 4 characteristics: effectiveness, productivity, safety, satisfaction. Sentences in user reviews were then manually matched with terminologies defined in the ontology. To classify polarity of sentences into positive, negative, or neutral, the approach used sentences labeled to be pros and cons in reviews and also used two lists of sentiment words to construct rule-based classifiers.

This paper is similar to these researches, especially Gunzmam and Laalej's work, since we aim to analyze user reviews of mobile applications to extract aspects and user sentiments about them. These researches however analyzed user reviews written in English whereas our work focuses on user reviews written in Thai. Processing Thai texts is more difficult since Thai texts need to be segmented into sentences and words. In addition, there is no lexical sentiment resource like SentiStrength for Thai language. 

\subsubsection{Research on mining of Thai texts} Several researches have done works on processing and mining Thai texts for various purposes. Three works are discussed here.

Inrak and Sinthupinyo \cite{emotioninthai} applied latent semantic analysis (LSA) \cite{LSA} to classify Thai texts from internet such as emails and blogs into six emotions: anger, disgust, fear, happiness, sadness, and surprise. They used SWATH \cite{SWATH} for word segmentation and ORCHID \cite{ORCHID} corpus to tag parts of speech.

In 2010, Haruechaiyasak et al. \cite{thaiopinionmininghotel} did opinion mining and sentiment analysis of hotel reviews written in Thai based on pre-determined features such as breakfast or service. From these features, related lexicons and a set of syntactic rules based on frequently occurred patterns were created to mine opinions and sentiments about these features.

In 2013, Haruechaiyasak et al. \cite{ssense} presented S-Sense, a framework for analyzing sentiments on Thai social media contents. The framework crawled and collected texts from social media such as Twitter and Pantip, a popular Thai webboard, and performed basic text processing such as word segmentation. It then classified texts into a predefined set of topics, and provided intention analysis to classify each text into four classes: announcement, request, question and sentiment. The sentiments were further classified into positive and negative. Their subsequent work \cite{ssense2} generalizes by extracting and analyzing keywords with statistical sigificance from social media contents.

All works that process Thai texts have similar text processing steps such as word segmentation and part-of-speech tagging. Inrak and Sinthupinyo's work \cite{emotioninthai} has a quite different purpose since it focuses on emotions while the other works \cite{thaiopinionmininghotel,ssense,ssense2} extract opinions and sentiments. Our work is more simlar to the latter. There are some differences however. Haruechaiyasak et al. \cite{thaiopinionmininghotel}'s work on hotel reviews has pre-determined features where ours dynamically extracts aspects from review texts. Lastly, the S-Sense framework \cite{ssense,ssense2} analyzes keywords and sentiments from social media. Ours is similar but we group keywords using LDA since some keywords can be grouped into similar aspects. 



