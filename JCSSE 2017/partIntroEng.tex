
Electronic devices especially smart phones and tablets are used so commonly nowadays that they could have been parts of our bodies. Mobile applications are therefore in demand and have been developed increasingly. To make mobile applications desirable, not only must developers follow their own visions, but they should also listen to user feedbacks to understand what users want in mobile applications.

By listening to user feedbacks and improve mobile applications accordingly, resulting applications would suit users better and even attract more users, otherwise users might stop using applications and in the end there might be no one using the applications at all. According to the survey \cite{145Q} that asked software engineers to rate and identify important questions about software engineering practices, the second most "essential" question software engineers wanted to know was "What parts of a software product are most used and/or loved by customer?" The answer to this question is very useful for developers in order to add, remove, or improve their application features and ultimately satisfy user needs.

One way to answer this question is to analyze user comments or reviews in application stores such as Apple App Store or Google Play Store. Users usually write reviews to praise or complain about mobile applications allowing other users to assess quality of the applications and enabling developers to get user feedbacks and improve application features. 

However, with so many user reviews, it is difficult or takes too much time to comprehend what users feel about mobile applications. Some comments might not be informative. For example, comments such as "ดี" (meaning "good") cannot tell specifically which part of the application is good. In addition, product rating cannot tell all the stories; it can only provide overall application quality but cannot give details which features users like or do not like.

This paper therefore aims to analyze user reviews written in Thai to automatically extract topics or features from the reviews and perform sentiment analysis to reveal which features users like or do not like. 

There are a number of related researches on automatic or semi-automatic opinion mining and sentiment analysis of user reviews on mobile applications \cite{ar-miner,userslikefeature,keywordmining,Leopairote1,Leopairote2}. More discussion on related work is presented in Section \ref{Related Work}.

The remaining of this paper is organized as follows. Section \ref{Related Work} describes related work in more details. Section \ref{Background} lays out theories needed for review analysis. Section \ref{Approach} presents our approach. Secton \ref{Result} discusses our results. Section \ref{Conclusion} provides conclusion and future works. 
%Please also note that this paper uses the word comments and reviews interchangeably.