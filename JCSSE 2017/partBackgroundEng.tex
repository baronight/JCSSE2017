Processing Thai texts to extract opinions and sentiments requires word segmentation, part of speech tagging, topic modeling, and sentiment analysis. This section describes these theories and researches that are basis for our work. 

\subsection{Word Segmentation}

Processing English text is easier than processing Thai text since English words in a sentence have spaces between them and therefore can be easily distinguished using a space as a delimiter. Thai text however is written continuously without spaces and difficult to find word boundaries. Therefore, there is a need for word segmentation when processing Thai text.

There are several word segmentation techniques such as Longest Matching\cite{syllableseparator}, Maximal Matching\cite{wordsegforthai}, Probabilistic Model\cite{thaiwordfilter}, and Feature-based Approach\cite{featurethaiwordseg}. The word segmentation tool used in this paper is LexTo\cite{LexTo} developed by National Electronics and Computer Technology Center (NECTEC) employs the Longest Matching technique. 

The Longest Matching technique attempts to match an entire string of Thai text with words contained in a given lexicon. If no word in the lexicon is matched with the string, the last character of the string is removed. The technique then attempts to match the truncated string with the lexicon again. The matching and truncating are repeated until a word is found in the lexicon. The word is then removed from the text, 
and the entire process is itereated for the entire text.



\subsection{Part of Speech}

In addition to word segmentation, we also need to tag each word a part of speech to identify whether this word is a (1) noun, (2) pronoun, (3) verb, (4) adverb, (5) adjective, (6) preposition, (7) conjunction, (8) interjection. The number of parts of speech can vary depending on how precisely one wants to tag.

Part-of-speech tagging is a common step in natural language processing. There are several tools that can help tag parts of speech such as Nature Language ToolKit (NLTK)\cite{NLTK} and RDRPOStagger\cite{RDRPOSTagger}. These tools need a corpus to learn parts of speech before being able to tag. NLTK has an English corpus and also allows any corpora from any languages to be added. RDRPOStagger has various corpora from seven languages including Thai. The Thai corpus in RDRPOStagger is called ORCHID\cite{ORCHID} developed by NECTEC. This paper uses the RDRPOStagger tool.

%NAiST\cite{NAiST} corpus developed by Department of Computer Engineering, Kasetsart University 


\subsection{Topic Modeling}

Opinion mining and sentiment analysis is a process that attempts to extract opinions and sentiments from given text\cite{surveyopinionmining}. For example, given user reviews of a product or service, one might want to know their opinions and feelings toward the product or service. In addition, one might want to know more in details which topics or aspects of the product or service users have opinions on. Topic modeling is a technique that helps discover underlying topics within given text. Latent Dirichlet allocation (LDA)\cite{LDA} is a common and effective topic modeling technique. LDA is based on a Bayesian model with the assumption that a given text contains a mixture of topics, and each topic contains various words with different probabilities.

There are various tools that can be used to analyze topic modeling. Our work uses ........ library in Python.

\subsection{Sentiment Analysis}

When extracting opinions about products or services, one also wants to know sentiments whether user opinions are positive, negative, or neutral. Two main approaches for sentiment analysis are machine learning based and lexicon-based. The machine-learning based approach learns and builds classifiers from texts with sentiments. The classifiers can then be used to classifiy sentiments. In the lexicion-based approach, each word has a sentiment score and sentiments of given text is calculated from scores of words appearing in the text. SentiWordNet~\cite{SentiWordNet} provides a lexical resource where English words are annotated with sentiment scores. Our work uses the SentiWordNet as a sentiment lexical resource.













